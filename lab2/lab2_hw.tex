% Options for packages loaded elsewhere
\PassOptionsToPackage{unicode}{hyperref}
\PassOptionsToPackage{hyphens}{url}
%
\documentclass[
]{article}
\usepackage{amsmath,amssymb}
\usepackage{lmodern}
\usepackage{iftex}
\ifPDFTeX
  \usepackage[T1]{fontenc}
  \usepackage[utf8]{inputenc}
  \usepackage{textcomp} % provide euro and other symbols
\else % if luatex or xetex
  \usepackage{unicode-math}
  \defaultfontfeatures{Scale=MatchLowercase}
  \defaultfontfeatures[\rmfamily]{Ligatures=TeX,Scale=1}
\fi
% Use upquote if available, for straight quotes in verbatim environments
\IfFileExists{upquote.sty}{\usepackage{upquote}}{}
\IfFileExists{microtype.sty}{% use microtype if available
  \usepackage[]{microtype}
  \UseMicrotypeSet[protrusion]{basicmath} % disable protrusion for tt fonts
}{}
\makeatletter
\@ifundefined{KOMAClassName}{% if non-KOMA class
  \IfFileExists{parskip.sty}{%
    \usepackage{parskip}
  }{% else
    \setlength{\parindent}{0pt}
    \setlength{\parskip}{6pt plus 2pt minus 1pt}}
}{% if KOMA class
  \KOMAoptions{parskip=half}}
\makeatother
\usepackage{xcolor}
\usepackage[margin=1in]{geometry}
\usepackage{color}
\usepackage{fancyvrb}
\newcommand{\VerbBar}{|}
\newcommand{\VERB}{\Verb[commandchars=\\\{\}]}
\DefineVerbatimEnvironment{Highlighting}{Verbatim}{commandchars=\\\{\}}
% Add ',fontsize=\small' for more characters per line
\usepackage{framed}
\definecolor{shadecolor}{RGB}{248,248,248}
\newenvironment{Shaded}{\begin{snugshade}}{\end{snugshade}}
\newcommand{\AlertTok}[1]{\textcolor[rgb]{0.94,0.16,0.16}{#1}}
\newcommand{\AnnotationTok}[1]{\textcolor[rgb]{0.56,0.35,0.01}{\textbf{\textit{#1}}}}
\newcommand{\AttributeTok}[1]{\textcolor[rgb]{0.77,0.63,0.00}{#1}}
\newcommand{\BaseNTok}[1]{\textcolor[rgb]{0.00,0.00,0.81}{#1}}
\newcommand{\BuiltInTok}[1]{#1}
\newcommand{\CharTok}[1]{\textcolor[rgb]{0.31,0.60,0.02}{#1}}
\newcommand{\CommentTok}[1]{\textcolor[rgb]{0.56,0.35,0.01}{\textit{#1}}}
\newcommand{\CommentVarTok}[1]{\textcolor[rgb]{0.56,0.35,0.01}{\textbf{\textit{#1}}}}
\newcommand{\ConstantTok}[1]{\textcolor[rgb]{0.00,0.00,0.00}{#1}}
\newcommand{\ControlFlowTok}[1]{\textcolor[rgb]{0.13,0.29,0.53}{\textbf{#1}}}
\newcommand{\DataTypeTok}[1]{\textcolor[rgb]{0.13,0.29,0.53}{#1}}
\newcommand{\DecValTok}[1]{\textcolor[rgb]{0.00,0.00,0.81}{#1}}
\newcommand{\DocumentationTok}[1]{\textcolor[rgb]{0.56,0.35,0.01}{\textbf{\textit{#1}}}}
\newcommand{\ErrorTok}[1]{\textcolor[rgb]{0.64,0.00,0.00}{\textbf{#1}}}
\newcommand{\ExtensionTok}[1]{#1}
\newcommand{\FloatTok}[1]{\textcolor[rgb]{0.00,0.00,0.81}{#1}}
\newcommand{\FunctionTok}[1]{\textcolor[rgb]{0.00,0.00,0.00}{#1}}
\newcommand{\ImportTok}[1]{#1}
\newcommand{\InformationTok}[1]{\textcolor[rgb]{0.56,0.35,0.01}{\textbf{\textit{#1}}}}
\newcommand{\KeywordTok}[1]{\textcolor[rgb]{0.13,0.29,0.53}{\textbf{#1}}}
\newcommand{\NormalTok}[1]{#1}
\newcommand{\OperatorTok}[1]{\textcolor[rgb]{0.81,0.36,0.00}{\textbf{#1}}}
\newcommand{\OtherTok}[1]{\textcolor[rgb]{0.56,0.35,0.01}{#1}}
\newcommand{\PreprocessorTok}[1]{\textcolor[rgb]{0.56,0.35,0.01}{\textit{#1}}}
\newcommand{\RegionMarkerTok}[1]{#1}
\newcommand{\SpecialCharTok}[1]{\textcolor[rgb]{0.00,0.00,0.00}{#1}}
\newcommand{\SpecialStringTok}[1]{\textcolor[rgb]{0.31,0.60,0.02}{#1}}
\newcommand{\StringTok}[1]{\textcolor[rgb]{0.31,0.60,0.02}{#1}}
\newcommand{\VariableTok}[1]{\textcolor[rgb]{0.00,0.00,0.00}{#1}}
\newcommand{\VerbatimStringTok}[1]{\textcolor[rgb]{0.31,0.60,0.02}{#1}}
\newcommand{\WarningTok}[1]{\textcolor[rgb]{0.56,0.35,0.01}{\textbf{\textit{#1}}}}
\usepackage{graphicx}
\makeatletter
\def\maxwidth{\ifdim\Gin@nat@width>\linewidth\linewidth\else\Gin@nat@width\fi}
\def\maxheight{\ifdim\Gin@nat@height>\textheight\textheight\else\Gin@nat@height\fi}
\makeatother
% Scale images if necessary, so that they will not overflow the page
% margins by default, and it is still possible to overwrite the defaults
% using explicit options in \includegraphics[width, height, ...]{}
\setkeys{Gin}{width=\maxwidth,height=\maxheight,keepaspectratio}
% Set default figure placement to htbp
\makeatletter
\def\fps@figure{htbp}
\makeatother
\setlength{\emergencystretch}{3em} % prevent overfull lines
\providecommand{\tightlist}{%
  \setlength{\itemsep}{0pt}\setlength{\parskip}{0pt}}
\setcounter{secnumdepth}{-\maxdimen} % remove section numbering
\ifLuaTeX
  \usepackage{selnolig}  % disable illegal ligatures
\fi
\IfFileExists{bookmark.sty}{\usepackage{bookmark}}{\usepackage{hyperref}}
\IfFileExists{xurl.sty}{\usepackage{xurl}}{} % add URL line breaks if available
\urlstyle{same} % disable monospaced font for URLs
\hypersetup{
  pdftitle={Lab 2 Homework},
  pdfauthor={Please Add Your Name Here},
  hidelinks,
  pdfcreator={LaTeX via pandoc}}

\title{Lab 2 Homework}
\author{Please Add Your Name Here}
\date{2023-01-12}

\begin{document}
\maketitle

\hypertarget{instructions}{%
\subsection{Instructions}\label{instructions}}

Answer the following questions and complete the exercises in RMarkdown.
Please embed all of your code and push your final work to your
repository. Your final lab report should be organized, clean, and run
free from errors. Remember, you must remove the \texttt{\#} for the
included code chunks to run. Be sure to add your name to the author
header above.

Make sure to use the formatting conventions of RMarkdown to make your
report neat and clean!

\begin{enumerate}
\def\labelenumi{\arabic{enumi}.}
\item
  What is a vector in R?
\item
  What is a data matrix in R?
\item
  Below are data collected by three scientists (Jill, Steve, Susan in
  order) measuring temperatures of eight hot springs. Run this code
  chunk to create the vectors.
\end{enumerate}

\begin{Shaded}
\begin{Highlighting}[]
\NormalTok{spring\_1 }\OtherTok{\textless{}{-}} \FunctionTok{c}\NormalTok{(}\FloatTok{36.25}\NormalTok{, }\FloatTok{35.40}\NormalTok{, }\FloatTok{35.30}\NormalTok{)}
\NormalTok{spring\_2 }\OtherTok{\textless{}{-}} \FunctionTok{c}\NormalTok{(}\FloatTok{35.15}\NormalTok{, }\FloatTok{35.35}\NormalTok{, }\FloatTok{33.35}\NormalTok{)}
\NormalTok{spring\_3 }\OtherTok{\textless{}{-}} \FunctionTok{c}\NormalTok{(}\FloatTok{30.70}\NormalTok{, }\FloatTok{29.65}\NormalTok{, }\FloatTok{29.20}\NormalTok{)}
\NormalTok{spring\_4 }\OtherTok{\textless{}{-}} \FunctionTok{c}\NormalTok{(}\FloatTok{39.70}\NormalTok{, }\FloatTok{40.05}\NormalTok{, }\FloatTok{38.65}\NormalTok{)}
\NormalTok{spring\_5 }\OtherTok{\textless{}{-}} \FunctionTok{c}\NormalTok{(}\FloatTok{31.85}\NormalTok{, }\FloatTok{31.40}\NormalTok{, }\FloatTok{29.30}\NormalTok{)}
\NormalTok{spring\_6 }\OtherTok{\textless{}{-}} \FunctionTok{c}\NormalTok{(}\FloatTok{30.20}\NormalTok{, }\FloatTok{30.65}\NormalTok{, }\FloatTok{29.75}\NormalTok{)}
\NormalTok{spring\_7 }\OtherTok{\textless{}{-}} \FunctionTok{c}\NormalTok{(}\FloatTok{32.90}\NormalTok{, }\FloatTok{32.50}\NormalTok{, }\FloatTok{32.80}\NormalTok{)}
\NormalTok{spring\_8 }\OtherTok{\textless{}{-}} \FunctionTok{c}\NormalTok{(}\FloatTok{36.80}\NormalTok{, }\FloatTok{36.45}\NormalTok{, }\FloatTok{33.15}\NormalTok{)}

\NormalTok{vecofvec }\OtherTok{\textless{}{-}} \FunctionTok{c}\NormalTok{(spring\_1, spring\_2, spring\_3, spring\_4, spring\_5, spring\_6, spring\_7, spring\_8)}

\NormalTok{matrox }\OtherTok{\textless{}{-}} \FunctionTok{matrix}\NormalTok{(vecofvec, }\AttributeTok{nrow =} \DecValTok{8}\NormalTok{, }\AttributeTok{byrow =}\NormalTok{ T)}

\NormalTok{matrox}
\end{Highlighting}
\end{Shaded}

\begin{verbatim}
##       [,1]  [,2]  [,3]
## [1,] 36.25 35.40 35.30
## [2,] 35.15 35.35 33.35
## [3,] 30.70 29.65 29.20
## [4,] 39.70 40.05 38.65
## [5,] 31.85 31.40 29.30
## [6,] 30.20 30.65 29.75
## [7,] 32.90 32.50 32.80
## [8,] 36.80 36.45 33.15
\end{verbatim}

\begin{Shaded}
\begin{Highlighting}[]
\FunctionTok{library}\NormalTok{(tidyverse)}
\end{Highlighting}
\end{Shaded}

\begin{verbatim}
## -- Attaching packages --------------------------------------- tidyverse 1.3.2 --
## v ggplot2 3.4.0      v purrr   1.0.1 
## v tibble  3.1.8      v dplyr   1.0.10
## v tidyr   1.2.1      v stringr 1.5.0 
## v readr   2.1.3      v forcats 0.5.2 
## -- Conflicts ------------------------------------------ tidyverse_conflicts() --
## x dplyr::filter() masks stats::filter()
## x dplyr::lag()    masks stats::lag()
\end{verbatim}

\begin{Shaded}
\begin{Highlighting}[]
\NormalTok{matrox}
\end{Highlighting}
\end{Shaded}

\begin{verbatim}
##       [,1]  [,2]  [,3]
## [1,] 36.25 35.40 35.30
## [2,] 35.15 35.35 33.35
## [3,] 30.70 29.65 29.20
## [4,] 39.70 40.05 38.65
## [5,] 31.85 31.40 29.30
## [6,] 30.20 30.65 29.75
## [7,] 32.90 32.50 32.80
## [8,] 36.80 36.45 33.15
\end{verbatim}

\begin{Shaded}
\begin{Highlighting}[]
\NormalTok{df\_matrox }\OtherTok{\textless{}{-}} \FunctionTok{as\_data\_frame}\NormalTok{(matrox)}
\end{Highlighting}
\end{Shaded}

\begin{verbatim}
## Warning: `as_data_frame()` was deprecated in tibble 2.0.0.
## i Please use `as_tibble()` instead.
## i The signature and semantics have changed, see `?as_tibble`.
\end{verbatim}

\begin{verbatim}
## Warning: The `x` argument of `as_tibble.matrix()` must have unique column names if
## `.name_repair` is omitted as of tibble 2.0.0.
## i Using compatibility `.name_repair`.
## i The deprecated feature was likely used in the tibble package.
##   Please report the issue at <]8;;https://github.com/tidyverse/tibble/issueshttps://github.com/tidyverse/tibble/issues]8;;>.
\end{verbatim}

\begin{enumerate}
\def\labelenumi{\arabic{enumi}.}
\setcounter{enumi}{3}
\item
  Build a data matrix that has the springs as rows and the columns as
  scientists.
\item
  The names of the springs are 1.Bluebell Spring, 2.Opal Spring,
  3.Riverside Spring, 4.Too Hot Spring, 5.Mystery Spring, 6.Emerald
  Spring, 7.Black Spring, 8.Pearl Spring. Name the rows and columns in
  the data matrix. Start by making two new vectors with the names, then
  use \texttt{colnames()} and \texttt{rownames()} to name the columns
  and rows.
\item
  Calculate the mean temperature of all eight springs.
\item
  Add this as a new column in the data matrix.
\item
  Show Susan's value for Opal Spring only.
\item
  Calculate the mean for Jill's column only.
\item
  Use the data matrix to perform one calculation or operation of your
  interest.
\end{enumerate}

\hypertarget{push-your-final-code-to-github}{%
\subsection{Push your final code to
GitHub!}\label{push-your-final-code-to-github}}

Please be sure that you check the \texttt{keep\ md} file in the knit
preferences.

\end{document}
